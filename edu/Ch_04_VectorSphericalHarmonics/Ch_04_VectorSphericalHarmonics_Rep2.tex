\documentclass[11pt]{article}

\usepackage{url}
\usepackage{hyperref}
\usepackage{graphicx}
\usepackage{verbatim}
\usepackage{color}
\setlength{\parskip}{0.5cm plus4mm minus3mm}
\usepackage{upquote}
\usepackage{float}

\textwidth=6.4in
\textheight=8.5in
\hoffset=-0.7in
\voffset=-0.7in

\setlength{\parindent}{0cm} 

\newcommand{\Yfun}{Y}
%\newcommand{\TAG}{test}
\newcommand{\TAG}{\begin{color}{blue}This tutorial is currently under construction. Please check back later for more by keeping your software updated.\end{color}}

\newcommand{\HERE}{\begin{color}{blue}Currently working on this part.\end{color}}

\hyphenation{Text-Wrangler}

\title{Chapter 4: Vector Spherical Harmonics}
\author{Kylee Ford, Sarah Kroeker, Alain Plattner}

\begin{document}
\maketitle

\section{Representation 2 of Vector Fields}

The second representation of vector fields include $E_{lm}$, $F_{lm}$, and $C_{lm}$.

$E_{lm}$: vector components from the gradient of a potential field from a planet. \\
$F_{lm}$: vector components from the gradient of a potential field from outside the satellite radius (space). \\
$C_{lm}$: same as in representation 1.

We can choose these values through the following method.  First, we will need to set some parameters:

\verb|L = 20;|\\
\verb|theta = pi/2;|\\
\verb|phi = pi;|

Now let's find the values of \verb|elm|:

\verb|[Elm,theta,phi] = elm(L,theta,phi);|

Similarly, we can find \verb|flm| and \verb|clm|:

\verb|[Flm,theta,phi] = flm(L,theta,phi);|\\
\verb|[Blm,Clm,theta,phi] = blmclm([],[],theta,phi);|

To plot the spherical harmonic coefficients, we must first convert each of the vector components into lmcosi format.  To do so, we can use the following:

\verb|elmcosi = coef2lmcosi(Elm,1) **DID NOT WORK|\\
\verb|flmcosi = fcoef2flmcosi(Flm,1);|\\
\verb|[blmcosi,clmcosi] = coef2blmclm(Clm,L);|

Now we can convert these to xyz coordinates by running:

\verb|[elm,elon,elat] = elm2xyz(elmcosi,1);|\\
\verb|[flm,flon,flat] = flm2xyz(flmcosi,1);|

The output of each of these provide fields \verb|elm{1}| (radial component), \verb|elm{2}| (theta or colatitudinal component), and \verb|elm{3}| (phi or longitudinal component).  This is also true for \verb|flm{i}|.  The first dimension of the field is latitude and the second is longitude.

\verb|[blmclm,lon,lat] = blmclm2xyz(blmcosi,clmcosi,1);|

This will output a field with \verb|blmclm(:,:,1)| as the phi component and \verb|blmclm(:,:,2)| as the theta component.  See the \verb|help| functions for each of these to examine their outputs in further detail.

Finally, we can plot these:

\verb|plotplm(elm,elon,elat,4,1)?|\\


If the vector field is represented as a linear combination of elm and flm, then we will need to evaluate elm and flm separately then sum them.



\TAG
\end{document}