\documentclass[11pt]{article}

\usepackage{url}
\usepackage{hyperref}
\usepackage{graphicx}
\usepackage{verbatim}
\usepackage{color}
\setlength{\parskip}{0.5cm plus4mm minus3mm}
\usepackage{upquote}
\usepackage{float}
\usepackage{subcaption}

\textwidth=6.4in
\textheight=8.5in
\hoffset=-0.7in
\voffset=-0.7in

\setlength{\parindent}{0cm} 

\newcommand{\Yfun}{Y}
%\newcommand{\TAG}{test}
\newcommand{\TAG}{\begin{color}{blue}This tutorial is currently under construction. Please check back later for more by keeping your software updated.\end{color}}

\newcommand{\HERE}{\begin{color}{blue}Currently working on this part.\end{color}}

\hyphenation{Text-Wrangler}

\title{Chapter 4: Vector Spherical Harmonics}
\author{Kylee Ford, Sarah Kroeker, Alain Plattner}

\begin{document}
\maketitle

\section{Representation 2 of Vector Fields}

The second representation of vector fields include $E_{lm}$, $F_{lm}$, and $C_{lm}$.

$E_{lm}$: vector components from the gradient of a potential field from a planet. \\
$F_{lm}$: vector components from the gradient of a potential field from outside the satellite radius (space). \\
$C_{lm}$: same as in representation 1.

\textbf{Exercise:} Let's calculate the spherical harmonics first for $E_{lm}$.  Let's set some parameters:

\verb|L = 2;|\\
\verb|theta = 0:0.01:pi;|\\
\verb|phi = 0:0.01:2*pi;|

We can now calculate them by running:

\verb|[E,theta,phi] = elm(L,theta,phi);|

This provides us with \verb|E|, 3 pixel maps of size $(L+1)^2$.  \verb|E{1}| are the radial components of the spherical harmonics, \verb|E{2}| are the colatitudinal components, and \verb|E{3}| are the longitudinal components.  Each row represents a \verb|(l,m)| pairing with columns for the pixel map values.  From the \verb|help| function, we can see that \verb|elm| function provides the \verb|m| values in the \textit{addmout} format.  

Let's plot the radial component of $E_{lm}$ with \verb|l=2| and \verb|m=-1|.  With the \textit{addmout} format, we will have this as the sixth row, which is denoted as \verb|E{1}(6,:)|.  Our first step would be to reshape this into a pixel matrix by running:

\verb|E6_1 = reshape(E{1}(6,:),length(theta),length(phi));|

Now we can plot this:

\verb|plotplm(E6_1,phi,pi/2-theta,2)|

We can also plot the colatitudinal component with the same (l,m).  We will need to reshape this as above and plot:

\verb|E6_2 = reshape(E{2}(6,:),length(theta),length(phi));|\\
\verb|plotplm(E6_2,phi,pi/2-theta,2)|

Plot the longitudinal component by running:

\verb|E6_3 = reshape(E{3}(6,:),length(theta),length(phi));|\\
\verb|plotplm(E6_3,phi,pi/2-theta,2)|

Compare your plots to figure 1.

\begin{figure}[H]
  \begin{subfigure}{.5\textwidth}
  \centering
  \includegraphics[width=0.5\textwidth]{figures_Rep2/E_2-1{1}.png}
  \caption{Radial component}
  \label{rad}
  \end{subfigure}
  \begin{subfigure}{.5\textwidth}
  \centering
  \includegraphics[width=0.5\textwidth]{figures_Rep2/E_2-1{2}.png}
  \caption{Colatitudinal component}
  \end{subfigure}
  \begin{subfigure}{.5\textwidth}
  \centering
  \includegraphics[width=0.5\textwidth]{figures_Rep2/E_2-1{3}.png}  
  \caption{Longitudinal component}
  \end{subfigure}
  \caption{Vector spherical harmonic with $E_{2-1}$ plot of the radial component in (a), colatitudinal component in (b), and the longitudinal component in (c).}
\label{E6}
\end{figure}

%%%%%%INSERT a figure with radial as a background and col/lon through quiver.

We can plot another (l,m), say $E_{20}$, which would be the the seventh row in $E_{lm}$.  Let's look at the radial component:

\verb|E7 = reshape(E{1}(7,:),length(theta),length(phi));|\\
\verb|plotplm(E7,phi,pi/2-theta,2)|

Compare your plot to figure 2.

\begin{figure}[H]
  \centering
  \includegraphics[width=0.5\textwidth]{figures_Rep2/E_20{1}.png}  
  \caption{Vector spherical harmonic with $E_{20}$ plot of the radial component.}
\label{E7}
\end{figure}

\subsection{Linear Combinations of Spherical Harmonics}

\textbf{Exercise:}  Let's try a linear combination, choose: $2E_{00}-1.5E_{10}+3E_{11}$.  We can plot this in a couple different ways, either by first reshaping each of the $E_{lm}$ values then make the linear combination or make the linear combination first then reshape it.  Try both and compare the results.  For the sake of computation, let's try the second way:

\verb|lincomb = 2*E{1}(1,:)-1.5*E{1}(3,:)+3*E{1}(4,:);|\\
\verb|linr = reshape(lincomb,length(theta),length(phi));|

Now we can plot:

\verb|plotplm(linr,[],[],2)|

Compare your plot to figure 3.

\begin{figure}[H]
  \centering
  \HERE
%  \includegraphics[width=0.5\textwidth]{figures_Rep2/lincomb??.png}  
  \caption{Linear combination of vector spherical harmonics, $2E_{00}-1.5E_{10}+3E_{11}$, with the radial component.}
\label{E7}
\end{figure}

We can also create the same linear combination by putting the values into \textit{addmout} format in the following way:

\verb|Elmcosi = [l,m,cosine expansion coefficient,sine expansion coefficient]|

The cosine and sine expansion coefficients are determined by whether m is positive, negative, or zero.  As described in Plattner and Simons (2017) in equation (2), we know that if m is negative or zero, the coefficient is the cosine portion and if m is positive, the coefficient is the sine portion.  So for our linear combination above, we can create a matrix:

\verb|Elmcosi = [0,0,2,0 ; 1,0,-1.5,0 ; 1,1,0,3]|



\HERE

If the vector field is represented as a linear combination of elm and flm, then we will need to evaluate elm and flm separately then sum them.



\TAG
\end{document}